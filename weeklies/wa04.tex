
\documentclass[12pt]{amsart}
\usepackage{geometry} % see geometry.pdf on how to lay out the page. There's lots.
\geometry{a4paper} % or letter or a5paper or ... etc
\usepackage[T1]{fontenc}
\usepackage[latin9]{inputenc}
\usepackage{amsmath}
\usepackage{amsaddr}
\usepackage{dirtytalk}
\usepackage{float}
\usepackage{listings}
\usepackage{hyperref}
\usepackage{enumerate}

\usepackage{color}
 
\definecolor{codegreen}{rgb}{0,0.6,0}
\definecolor{codegray}{rgb}{0.5,0.5,0.5}
\definecolor{stringcolor}{rgb}{0.7,0.23,0.36}
\definecolor{backcolour}{rgb}{0.95,0.95,0.92}
\definecolor{keycolor}{rgb}{0.007,0.01,1.0}
\definecolor{itemcolor}{rgb}{0.01,0.0,0.49}

\hypersetup{
    colorlinks=true,
    linkcolor=blue,
    filecolor=blue,      
    urlcolor=blue,
}
 
\lstdefinestyle{mystyle}{
    %backgroundcolor=\color{backcolour},   
    commentstyle=\color{codegreen},
    keywordstyle=\color{keycolor},
    numberstyle=\tiny\color{codegray},
    stringstyle=\color{stringcolor},
    basicstyle=\footnotesize,
    breakatwhitespace=false,         
    breaklines=true,                 
    captionpos=b,                    
    keepspaces=true,                 
    numbers=left,                    
    numbersep=5pt,                  
    showspaces=false,                
    showstringspaces=false,
    showtabs=false,                  
    tabsize=2
}
 
\lstset{style=mystyle}

\lstdefinelanguage{Swift}{
  keywords={associatedtype, class, deinit, enum, extension, func, import, init, inout, internal, let, operator, private, protocol, public, static, struct, subscript, typealias, var, break, case, continue, default, defer, do, else, fallthrough, for, guard, if, in, repeat, return, switch, where, while, as, catch, dynamicType, false, is, nil, rethrows, super, self, Self, throw, throws, true, try, associativity, convenience, dynamic, didSet, final, get, infix, indirect, lazy, left, mutating, none, nonmutating, optional, override, postfix, precedence, prefix, Protocol, required, right, set, Type, unowned, weak, willSet},
  ndkeywords={class, export, boolean, throw, implements, import, this},
  sensitive=false,
  comment=[l]{//},
  morecomment=[s]{/*}{*/},
  morestring=[b]',
  morestring=[b]"
}

\lstset{emph={Int,count,abs,repeating,Array}, emphstyle=\color{itemcolor}}


\title{Week 03}

\date{\today}

\lstset{style=mystyle}

%%% BEGIN DOCUMENT
\begin{document}
\maketitle

\section{Preparation for Assignment}
If, and \textit{only if} you can truthfully assert the truthfulness of each statement below are you ready to start the assignment.
\subsection {Reading Comprehension Self-Check}
\begin{itemize}
\item  I know that \textit{decrease-and-conquer} is a general algorithm design technique based on exploiting a relationship between a solution to a given instance of a problem and a solution to a smaller instance of the same problem.
\item  I can give a common example of each of the three major variations of the \textit{decrease-and-conquer} technique.
\item  I know that \textit{insertion-sort}'s notable advantage is good performance on almost-sorted arrays.
\item  I know that the \textit{topological sorting problem} for a directed graph has a solution if and only if the graph has no directed cycles.
\item  I know that an application of topological sorting is resolving symbol dependencies in linkers.
\item  I know that a \textit{dag} is a directed acyclic graph.
\item  I know that the \textit{decrease-by-one} technique is a natural approach to developing algorithms for generating elementary combinatorial objects.
\item  I know that \textit{binary search} is the most important and well-known example of a \textit{decrease-by-a-constant-factor} algorithm.
\item  I know why the Euclidean GCD algorithm is an excellent example of a \textit{decrease-by-a-variable-size} algorithm.
\item  I know how to order and have ordered the seven Big-$\mathcal{O}$ efficiency classes shown below from \textit{fastest growing} reference function (first) to \textit{slowest growing} reference function (last):
    \begin{enumerate}
    \item $\mathcal{O}(\log n)$
    \item  $\mathcal{O}(n!)$
    \item  $\mathcal{O}(n)$
    \item  $\mathcal{O}(1)$
    \item  $\mathcal{O}(2^n)$
    \item  $\mathcal{O}(n \log n)$
    \item  $\mathcal{O}(n^2)$
   \end{enumerate}
\item  I know given a $\mathcal{O}(n^3)$ algorithm whose running time for a problem of size 100,000 is 10 seconds what its running time would be for a problem of size 150,000.
\item  I know given a $\mathcal{O}(1)$ algorithm whose running time for a problem of size 100,000 is 10 seconds what its running time would be for a problem of size 200,000.

\end{itemize}
\subsection{Memory Self-Check}

\subsubsection{Algorithm Efficiency Calculation}


\begin{enumerate}
     \item Code and understand a \textit{reduce-by-variable-size} algorithm, without looking at pseudocode, that is an implementation of Euclid's GCD algorithm.
     \item Write a non-bruite-force algorithm to generate all subsets of a set (A power set).
     \item Write a non-bruite-force algorithm that solves the Josephus problem.
\end{enumerate}.



\section{Week 04 Exercises}
\subsection{ Exercise 7 on page 137} 
\subsection{Exercise 1 on page 142} 
\subsection{Exercise 2 on page 148} 
\subsection{Exercise 1 on page 156}
\subsection{ Exercise 1 on page 166} 
\subsection{Exercise 12 on page 167} 


\section{Week 04 Problems}
\subsection{Exercise 6 on page 137} 
\subsection{Exercise 10 on page 143} Make sure you write out all the mathematical steps to get the result.
\subsection{Exercise 12 on page 149}
\subsection{Exercise 8 on page 156}


\end{document}